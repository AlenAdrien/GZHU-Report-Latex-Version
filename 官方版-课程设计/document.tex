\documentclass[10pt,a4paper]{article}
\usepackage{graphicx}
\usepackage{ctex}
\usepackage{indentfirst}
%\graphicspath{{chapter/}{figures/}}
\usepackage{CJK}
\usepackage{amsmath}%数学


\usepackage{makecell}%表格竖线连续
\newcommand\toprule{\Xhline{.08em}}
\newcommand\midrule{\Xhline{.05em}}
\newcommand\bottomrule{\Xhline{.08em}}
\def\I{\vrule width1.2pt}
%!\I 就可以代替| 来画表格了

%可固定下划线长度
\makeatletter
\newcommand\dlmu[2][4cm]{\hskip1pt\underline{\hb@xt@ #1{\hss#2\hss}}\hskip3pt}
\makeatother


\usepackage{array}%数学
\usepackage{multirow}%跨行表格
%\usepackage[colorlinks,linkcolor=red]{hyperref}%超链接

\usepackage{fancyhdr}  %使用fancyhdr包自定义页眉页脚
%\pagestyle{empty}
\pagestyle{fancy}
%\pagestyle{plain}%没有页眉,页脚放页数
\renewcommand{\headrulewidth}{0.5pt}
\renewcommand{\footrulewidth}{0.4pt}
\lhead{}
\chead{}
\rhead{}
\lfoot{}
\cfoot{\thepage}
\rfoot{}


\usepackage{booktabs}%表格用

\usepackage{float}%可以用于禁止浮动体浮动

%目录超链接
\usepackage[colorlinks,linkcolor=black,anchorcolor=blue,citecolor=black]{hyperref}

\usepackage{listings}%可以插入代码
\usepackage{xcolor}%语法高亮支持
%代码格式
\definecolor{dkgreen}{rgb}{0,0.6,0}
\definecolor{gray}{rgb}{0.5,0.5,0.5}
\definecolor{mauve}{rgb}{0.58,0,0.82}
\lstset{ %
	%	language=Python,                % the language of the code
	breaklines,%自动折行
	%extendedchars=false%解决代码跨页时,章节标题,页眉等汉字不显示的问题
	keepspaces=false,  
	%tabsize=4 %设置tab空格数
	showspaces=false,  %不显示空格
	showtabs=false,  
	showstringspaces=true, 
	numbers=left, 
	basicstyle=\footnotesize, 
	numberstyle=\tiny, 
	numbersep=5pt, 
	keywordstyle= \color{ blue!70},%关键字颜色
	commentstyle= \color{red!50!green!50!blue!50},%注释颜色 
	frame=shadowbox, % 边框格式:阴影效果
	rulesepcolor= \color{ red!20!green!20!blue!20} ,
	escapeinside=``, % 英文分号中可写入中文
	xleftmargin=2em,xrightmargin=2em, aboveskip=1em,%设置页边距
	framexleftmargin=2em
}




%设置页面格式
\usepackage[left=2.0cm, right=2.0cm, top=2.0cm, bottom=2.0cm]{geometry}
\begin{document}
%%%%%%%%%%%%%%%%%%%%%%%%%%%%%%%%%%%%%%
%%%%%%%%%%%%
%%%%%%%%%%%%封面
%%%%%%%%%%%%
%%%%%%%%%%%%%%%%%%%%%%%%%%%%%%%%%%%%%%
\begin{titlepage}
	
\centering
{\huge{\quad 广州大学 \quad 机械与电气工程学院\par}}
{\Huge\kaishu{\quad 电子与通信工程系\par}}
\vspace{1cm}
{\Huge
	课\par
	程\par
	设\par
	计\par
	报\par
	告\par
}

\vspace{1cm}

\begin{flushleft}
	{\fangsong\Large \qquad\qquad\qquad 课程名称:\dlmu[8cm]{电子系统课程设计 }\par}
	\vspace{0.1cm}
	{\fangsong\Large \qquad\qquad\qquad 设计题目:\dlmu[8cm]{基于 某某 的 某某}\par}
	\vspace{.1cm}
	{\fangsong\Large \qquad\qquad\qquad 专业班级:\dlmu[5cm]{电子信息工程}{15级} \dlmu[1cm]{1}{班}\par}
	\vspace{.1cm}
	{\fangsong\Large \qquad\qquad\qquad 组长姓名:\dlmu[1.8cm]{某某某}学号\dlmu[3cm]{1407400106}成绩\dlmu[1cm]{}\par}
	\vspace{.1cm}
	{\fangsong\Large \qquad\qquad\qquad 成员姓名:\dlmu[1.8cm]{某某}学号\dlmu[3cm]{1507400052}成绩\dlmu[1cm]{}\par}
	\vspace{.1cm}
	{\fangsong\Large \qquad\qquad\qquad 成员姓名:\dlmu[1.8cm]{某某某}学号\dlmu[3cm]{1507400056}成绩\dlmu[1cm]{}\par}
	\vspace{.1cm}
	{\fangsong\Large \qquad\qquad\qquad 成员姓名:\dlmu[1.8cm]{某某某}学号\dlmu[3cm]{1507400051}成绩\dlmu[1cm]{}\par}
	\vspace{.1cm}
	{\fangsong\Large \qquad\qquad\qquad 指导老师:\dlmu[8.1cm]{老师姓名}\par}
	\vspace{.1cm}
	{\fangsong\Large \qquad\qquad\qquad 所在学期:\dlmu[8.1cm]{2018-2019学年第一学期}\par}
	\vspace{.1cm}
	{\fangsong\Large \qquad\qquad\qquad 所在时间:\dlmu[8.1cm]{2018年12月17-28日}\par}
	\vspace{.1cm}
	{\fangsong\Large \qquad\qquad\qquad 地\qquad 点:\dlmu[8.1cm]{电子信息实验楼510}\par}
\end{flushleft}

\end{titlepage}
\tableofcontents
\newpage






%%%%%%%%%%%%%%%%%%%%%%%%%%%%%%%%%%%%%%
%%%%%%%%%%%%
%%%%%%%%%%%%双并列图片示例
%%%%%%%%%%%%
%%%%%%%%%%%%%%%%%%%%%%%%%%%%%%%%%%%%%%
%\begin{figure}[H]
%	\centering
%	\begin{minipage}[t]{0,40\textwidth}	
%		\centering
%		\includegraphics[scale=0.5]{ccjg.pdf} %1.png是图片文件的相对路径
%		\caption{IEEE 802.11层次结构} %caption是图片的标题
%		\label{p_ccjg} %此处的label相当于一个图片的专属标志,目的是方便上下文的引用
%	\end{minipage}
%	\hfil
%	\begin{minipage}[t]{0,50\textwidth}	
%		\centering
%		\includegraphics[scale=1]{AODV.pdf} %1.png是图片文件的相对路径
%		\caption{AODV示意图} %caption是图片的标题
%		\label{p_AODV} %此处的label相当于一个图片的专属标志,目的是方便上下文的引用
%	\end{minipage}
%\end{figure}


%%%%%%%%%%%%%%%%%%%%%%%%%%%%%%%%%%%%%%
%%%%%%%%%%%%
%%%%%%%%%%%%表格示例
%%%%%%%%%%%%
%%%%%%%%%%%%%%%%%%%%%%%%%%%%%%%%%%%%%%

% Table generated by Excel2LaTeX from sheet 'Sheet1'
%\begin{table}[H]
%	\centering
%	\caption{802.11a/b/g物理层,MAC层参数}
%	\begin{tabular}{ccccc}
%		\toprule
%		&  参数  & 802.11a & 802.11b & 802.11g \\
%		\midrule
%		\multirow{4}[7]{*}{物理层} & 频带/Hz(freq\_) & $5*10^9$ & $2.4*10^9$ & $2.4*10^9$ \\
%		\cmidrule{3-5}       & 通信感知范围\cite{bib13}(CSThresh\_) & $3.17291*10^9$ & $2.79*10^9$ & $2.79*10^9$ \\
%		\cmidrule{3-5}       & 可通信范围\cite{bib13}(RXThresh\_) & $6.5556*10^{10}$ & $5.76*10^9$ & $5.76*10^9$ \\
%		\cmidrule{3-5}       & 传输功率/W(Pt\_) & 0.281838 & 0.281838 & 0.281838 \\
%		\midrule
%		\multirow{9}[17]{*}{MAC层} & 竞争窗口最小值\cite{bib12}/s(CWMin) & 15 & 31 & 15 \\
%		\cmidrule{3-5}       & 竞争窗口最大值\cite{bib12}/s(CWMax) & 1023 & 1023 & 1023 \\
%		\cmidrule{3-5}       & 时隙\cite{bib11}/s(SlotTime\_) & 0.00005 & 0.00002 & 0.000009s \\
%		\cmidrule{3-5}       & SIFS\cite{bib14}\cite{bib11}/s(SIFS\_) & 0.000016 & 0.00001 & 0.000016s \\
%		\cmidrule{3-5}       & 前导码长度\cite{bib14}(PreambleLength) & 96 & 144 & 120 \\
%		\cmidrule{3-5}       & PLCP头部长度\cite{bib14}PLCPHeaderLength\_) & 24 & 48 & 24 \\
%		\cmidrule{3-5}       & PLCP数据率\cite{bib14}/bps(PLCPDataRate\_) & $6*10^6$ & $1*10^6$ & $6*10^6$ \\
%		\cmidrule{3-5}       & 最高速率\cite{bib14}/bps(dataRate) & $5.4*10^7$ & $1.1*10^7$ & $5.4*10^7$ \\
%		\cmidrule{3-5}       & 最低速率\cite{bib14}/bps(basicRate\_) & $6*10^6$ & $1*10^6$ & $6*10^6$ \\
%		\bottomrule
%	\end{tabular}%
%	\label{t_abgcs}%
%\end{table}%

%%%%%%%%%%%%%%%%%%%%%%%%%%%%%%%%%%%%%%
%%%%%%%%%%%%
%%%%%%%%%%%%插入代码示例
%%%%%%%%%%%%title:代码文件标题
%%%%%%%%%%%%basicstyle:字体大小
%%%%%%%%%%%%language:语言,C++,C,Matlab,Python
%%%%%%%%%%%%%%%%%%%%%%%%%%%%%%%%%%%%%%
%\lstset{language=C++}
%\begin{lstlisting}[title=AODV100.tr,basicstyle=\tiny]
%
%\end{lstlisting}
\section{背景介绍}

\section{原理}

%%%%%%%%%%%%%%%%%%%%%%%%%%%%%%%%%%%%%%
%%%%%%%%%%%%
%%%%%%%%%%%%对齐公式示例
%%%%%%%%%%%%
%%%%%%%%%%%%%%%%%%%%%%%%%%%%%%%%%%%%%%

%\begin{align}
%	\label{kk}
%	k&=\dfrac{3Z_{11}^{'}}{2(1-l^2_2)^{3/2}}\\
%	\label{hh}
%	h&=\frac{1}{\pi}\left[Z_{00}-\frac{k\pi}{2}+k\arcsin(l_2)+kl_2\sqrt{1-l^2_2} \right]\\
%	\label{ll} l&=\frac{1}{2}\left[\sqrt{\frac{5Z_{40}^{'}+3Z^{'}_{20}}{8Z_{20}}}+\sqrt{\frac{5Z_{11}^{'}+Z^{'}_{11}}{6Z_{11}}}\right]\\
%	\label{pp}
%	\phi&=\arctan\left[\frac{Im[Z_{n1}]}{Re[Z_{n1}]}\right]
%\end{align}







%%%%%%%%%%%%%%%%%%%%%%%%%%%%%%%%%%%%%%
%%%%%%%%%%%%
%%%%%%%%%%%%表格示例2
%%%%%%%%%%%%
%%%%%%%%%%%%%%%%%%%%%%%%%%%%%%%%%%%%%%
%\begin{table}[H]
%	\centering
%	\caption{NVIDIA$^{\textregistered}$ Jetson TK1配置一览}
%	\vspace{0.5cm}
%	\begin{tabular}{l}
%		\Xhline{1.2pt}
%		Tegra K1 SOC \\
%		NVIDIA$^{\textregistered}$ Kepler$^{\textregistered}$ GPU、192 个 CUDA 核心 \\
%		NVIDIA$^{\textregistered}$ 4-Plus-1™ 四核 ARM$^{\textregistered}$ Cortex™-A15 CPU \\
%		2 GB x16 内存、64 位宽度 \\
%		16 GB 4.51 eMMC 内存 \\
%		1 个 USB 3.0 端口、A  \\
%		1 个 USB 2.0 端口、Micro AB\\
%		1 个半迷你 PCIE 插槽\\
%		1 个完整尺寸 SD/MMC 连接器\\
%		1 个 RTL8111GS Realtek 千兆位以太网局域网 \\
%		1 个 SATA 数据端口 \\
%		1 个完整尺寸 HDMI 端口 \\
%		1 个 RS232 串行端口 \\
%		SPI 4 兆字节引导闪存\\
%		1 个带 Mic In 和 Line Out 的 ALC5639 Realtek 音频编解码器\\
%		以下信号可通过扩展端口获得:DP/LVDS, Touch SPI 1x4 + 1x1 CSI-2, GPIOs, UART, HSIC, I$^2$C
%		\\
%		\Xhline{1.2pt}
%	\end{tabular}%
%	\label{aaa}%
%\end{table}%

%%%%%%%%%%%%%%%%%%%%%%%%%%%%%%%%%%%%%%
%%%%%%%%%%%%
%%%%%%%%%%%%双并列表格示例
%%%%%%%%%%%%
%%%%%%%%%%%%%%%%%%%%%%%%%%%%%%%%%%%%%%
%\begin{table}[H]\footnotesize
%	\centering
%	
%	\begin{minipage}[t]{0,47\textwidth}		
%		\caption{上位机配置清单}
%		\vspace{0.5cm}
%		\centering
%		\begin{tabular}{cc}
%			\Xhline{1.2pt}
%			运行环境 & ubuntu14 (基于Cortex$^{\textregistered}$-A15芯片) \\
%			编程语言 & C/C++ \\
%			第三方库及组件 & GTK2.0,OpenCV2.4.10 \\
%			开发环境 & Qt Creator 与 make工程管理器  \\
%			编译工具链 & NVIDIA$^{\textregistered}$-ARM$^{\textregistered}$编译工具链 \\
%			程序结构 & 模块化结构 \\
%			\Xhline{1.2pt}
%		\end{tabular}%
%		
%		\label{pzqd}%
%	\end{minipage}
%	\hfil
%	\hfil
%	\begin{minipage}[t]{0,47\textwidth}	
%		\centering
%		\caption{上位机功能清单}
%		\vspace{0.5cm}	
%		\begin{tabular}{cc}
%			\Xhline{1.2pt}
%			编号  & \multicolumn{1}{c}{功能描述} \\
%			\Xhline{1.2pt}
%			1   & \multicolumn{1}{c}{可打开/关闭摄像头} \\
%			2   & 可通过摄像头捕获图片为目标图片 \\
%			3   & 可从文件系统内选择图片并载入为目标图片 \\
%			4   & 可以检测目标图片中圆形轮廓的半径和圆心 \\
%			5   & 可以检测目标图片中平行直线的间距 \\
%			6   & 检测算法的参数可自由调整 \\
%			\Xhline{1.2pt}
%		\end{tabular}%
%		\label{gn}%
%	\end{minipage}
%\end{table}%




\section{实验}
\subsection{平台与配置}

实验平台:NVIDIA$^{\textregistered}$ Jetson TK1板,$Opencv2.4.10$,$GTK2.0$

\subsection{实验结果与分析}



\section{总结}

\begin{thebibliography}{4}
\bibitem{bib1}Ghosal S, Mehrotra R. Orthogonal moment operators for subpixel edge detection. Pattern recognition, 1993, 26(2): 295-306.

\end{thebibliography}


\end{document}
